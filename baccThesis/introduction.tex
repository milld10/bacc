\section*{Introduction}
\label{introduction}

Studies show that the average email address is linked to over 100 accounts. \cite{DigitalGuardian} 
When generating and storing personal passwords, the average user does it in an unsafe manner. \cite{pilar2012passwords} As discussed in \cite{DBLP:journals/cacm/AdamsS99}, the average user also does not create particularly safe or hard to remember passwords. Often users reuse the same or similar password for multiple accounts. \cite{DBLP:journals/cacm/AdamsS99} \cite{pilar2012passwords} \cite{Statista}

Password managers can help create and remember different passwords more easily. There are different kinds of password managers. Some store the data locally, for example on a computer or mobile device. This is a safe option to manage credentials. However, making them accessible throughout different devices can be challenging. To make them available on multiple devices, some password managers offer cloud synchronization. 
This provides the shared infrastructure to consumers. Privacy issues may occur if sensitive information can be retrieved, such as, unauthorized secondary usage, lack of user control and unclear resonsibility. \cite{DBLP:journals/corr/SainiM14} Then there are password managers that are entirely cloud-based, meaning that no data is saved locally on any device and credentials are accessed over a website. Additionally, browser integration may be supported, which makes passwords accessible through the browser in use.

The goal is to provide secure and independent credential storage. To reduce external dependencies, no cloud services shall be used. Credentials are accessible on multiple devices via a secure Bluetooth connection. In this project, an Android credential manager is designed.  Additionally, we developed a Google Chrome extension, which uses the Web Bluetooth API \cite{WebBTAPI}. The extension and application establish a secured Bluetooth Low Energy (LE) connection to transfer the credentials. \\
Data is stored on the device. On demand one selected credential can be shared with the extension at a time. Accessing and sending credentials requires authentication from the user via fingerprint. This process ensures that only registered fingerprints can access and share sensitive data. 
The Chrome extension then fills username and password in into the text fields on the website. \\

This thesis consists of 9 chapters that are structured as follows: Chapter \ref{motivation} shows the motivation behind this project, while section \ref{relwork} explains related work that has already been done on this subject. In Chapter \ref{background}, the background of technologies used is described further. Chapter \ref{architecture} handles the architecture of the project and presents the topics in detail. Section \ref{seceval} illustrates a brief security evaluation based on common criteria, and Chapter \ref{bluetooth} discusses Bluetooth, and it's underlying security. Furthermore, Chapter \ref{limitations} describes the limitations that apply to this project and future work is explained in Chater \ref{futurework}. Finally, the conclusions are drawn in Chapter \ref{conclusion}.

