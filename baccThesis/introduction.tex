\section*{Introduction}
\label{introduction}

%Introduction into the topic; what was made; what are the problems with state of the art technology, generally a description of what was made, more like a story

Nowadays password managers are essential to our everyday lives. Studies show that the average person has up to 20 online accounts [paper]. As shown in [ref:typicalpassword], the average user also does not create particularly safe or hard to remember passwords. Often the same password or some variation of it is reused over multiple different accounts. These passwords may be vulnerable to attacks. [ref] \\

To maintain a large number of logins, many users rely on password managers. This approach allows to create, and remember different passwords. There are different kinds of password managers. Some store the data locally, for example on a computer or mobile device. This is a safe option to manage credentials. However, making them accessible throughout different devices can be challenging. To make them available on multiple devices, some password managers offer cloud synchronization. This may be very convenient when working on different devices. Nevertheless, the risk of data being corrupted, stolen or wiped is still given. [https://www.bbc.com/news/business-36151754] [paper attack on cloud services]. Then there are password managers that are entirely cloud-based, meaning that no data is saved locally on any device and credentials are accessed over a website. Also, some password managers support browser integration, which makes passwords accessible through the browser in use. By uploading sensitive data in any form onto the Internet, the risk of an attack may increase. [ref] \\

The goal is to provide secure and independent credential storage. To reduce external dependencies, no cloud services shall be used. Credentials are accessible on multiple devices via a secure Bluetooth connection. In this project, an Android credential manager is designed.  Additionally, we developed a Google Chrome extension, which uses the Web Bluetooth API [ref web bt api]. The extension and application establish a secured Bluetooth Low Energy (LE) connection to transfer the credentials. \\
Data is stored on the device. On demand one selected credential can be shared with the extension at a time. Accessing and sending credentials requires authentication from the user via fingerprint. This process ensures that only registered fingerprints can access and share sensitive data. 
The Chrome extension [chrome API link] then fills username and password in into the text fields on the website. \\ 
%https://developers.google.com/web/updates/2015/07/interact-with-ble-devices-on-the-web)

This thesis consists of 8 chapters that are structured as follows: Chapter \ref{motivation} shows the motivation behind this project, while section \ref{relwork} explains related work that has already been done on this subject. In Chapter \ref{background}, the background of technologies used is described further. Chapter \ref{architecture} handles the architecture of the project and presents the topics in detail. Section \ref{seceval} illustrates a brief security evaluation based on common criteria, and Chapter \ref{bluetooth} discusses Bluetooth, and it's underlying security. Furthermore, Chapter \ref{limitations} describes the limitations that apply to this project and future work is explained in Chater \ref{futurework}. Finally, the conclusions are drawn in Chapter \ref{conclusion}.

