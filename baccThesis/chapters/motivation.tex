\section{Motivation} \label{motivation}
There exist many applications that offer services to store user credentials securely. Most users work on more than one device, which makes the availability of credentials important.
By synchronizing data with cloud services, credentials are made available more easily. However, using cloud services can compromise data confidentiality. \cite{AndroidPhishing} states that the Android application of LastPass is prone to exploitation through, for example, phishing attacks. Also, as shown in \cite{PMLeak}, in LastPass, as well as other well-known managers that offer cloud synchronization, security vulnerabilities were found that compromise confidentiality of data. However, if cloud services are renounced, availability on different devices can be challenging.

Mobile phones support our everyday lives and can help to make data available wherever the user is. Additionally, they offer a secure way to store data. For example, most Android phones rely on the hardware-backed key storage. Therefore, secure data storage for user credentials can be provided. 

The motivation for this project is to provide secure storage and availability while reducing external dependencies. By storing them on a mobile device, authenticity can be provided. In Section \ref{arch_keystore} we discuss how the smartphone hardware can provide a secure environment for data storage. To simplify the distribution of credentials we use a Bluetooth connection between a mobile device and computer in use. This allows credentials to be available throughout multiple devices. Through the Bluetooth communication channel, data is sent to the computer and filled into the login forms. Hence, third parties are excluded, and confidentiality can be provided. Since data is not saved on the computer, the risk of threatened security is reduced. Simultaneously, it is possible to retrieve credentials from a different computer. 

