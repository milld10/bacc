\section{Introduction} \label{introduction}
A study \cite{DigitalGuardian} shows that the average email address is linked to over 100 accounts.
When generating and storing personal passwords, the average user does it in an unsafe manner as stated in \cite{pilar2012passwords}. As discussed in \cite{AdamsS99}, the average user also does not create particularly safe or hard to remember passwords. As mentioned in \cite{AdamsS99}, \cite{pilar2012passwords}, and \cite{Statista}, users often reuse the same password for multiple accounts.

Password managers can help create and remember different passwords more easily. There exist different kinds of password managers. Local password managers store data locally, for example on a computer or mobile device. This is a safe option to manage credentials. However, making them available throughout different devices can be challenging. \\
Then there are web-based password managers, meaning that no data is saved locally on any device and credentials are retrieved over a website. Additionally, browser integration may be supported, which makes passwords available through the browser in use. \\
A cloud synchronization may be offered in both kinds of password managers to make them available on multiple devices. However, this provides the shared infrastructure to consumers. \cite{SainiM14} discusses that privacy issues may occur if sensitive information can be retrieved, such as, unauthorized secondary usage, lack of user control and unclear resonsibility.

There exist countless options to store sensitive data. In this work, we want to explore the possibility of making credentials available without depending on third party services. Our goal is to provide secure and independent credential storage. To reduce external dependencies, no cloud services shall be used. Simultaniously credentials remain available on multiple devices via a Bluetooth connection. \\
To show the feasibility, we designed and implemented an Android credential manager.  Additionally, we developed a acompaning Google Chrome extension, which uses the Web Bluetooth API \cite{WebBTAPI}. The extension and application establish a secured \gls{ble} connection to transfer the credentials. \\
Data is stored on the device. On demand one selected credential can be shared with the extension at a time. Accessing and sending credentials requires authentication from the user via fingerprint. This process ensures that only registered fingerprints can access and share sensitive data. 
The Chrome extension then fills username and password in into the forms on the website. \\

This thesis consists of 9 chapters that are structured as follows: Chapter \ref{motivation} shows the motivation behind this project, while section \ref{relwork} explains related work that has already been done on this subject. In Chapter \ref{background}, the background of technologies used is described further. Chapter \ref{architecture} handles the architecture of the project and presents the topics in detail. Section \ref{seceval} illustrates a brief security evaluation based on common criteria, and Chapter \ref{bluetooth} discusses Bluetooth, and it's underlying security. Furthermore, Chapter \ref{limitations} describes the limitations that apply to this project and future work is explained in Chater \ref{futurework}. Finally, the conclusions are drawn in Chapter \ref{conclusion}.

