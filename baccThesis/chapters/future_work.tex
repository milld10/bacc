\section{Future Work} \label{futurework}
This Section presents ideas that could be realized in the near future once the necessary technology has progressed. Also, we discuss improvements regarding usability, convenience, and security of the Bluetooth connection, the extension, and the application. These features are not part of the scope of this project and therefore have not been implemented.

\subsection{Improvements on the Bluetooth connection}
One possible future implementation can be the end-to-end encryption of the \gls{ble} data streams to provide further security. By providing end-to-end encryption, a more secure environment to exchange credentials can be provided. Therefore, the implementation of classic Bluetooth can be spared by increasing the security of the protocol. As mentioned in Section \ref{bluetooth}, with \gls{ble} the risk of extracting data via packet sniffers exists. By securing the data stream with an end-to-end encryption, retrieving data in plaintext is made challenging. \\

Another consideration is switching from a \gls{ble} to a classic Bluetooth connection. This depends on Google's Web Bluetooth \gls{api}. Currently only \gls{ble} is supported. As discussed in Section \ref{bluetooth}, \gls{ble} can lead to a vulnerable communication channel when keys can be retrieved during the pairing process. Also, the Web Bluetooth \gls{api} is still experimental, and the specification is not final yet. This leaves the possibility open for further development and eventually the support for classic Bluetooth.
 
\subsection{Improvements of the Chrome extension}
In \cite{VarshneyBS18} malicious extensions are discussed and how a Chrome extension can become vulnerable to various attacks. Malicious extensions can perform attacks, where sensitive data is phished from the user or keystrokes are recorded to recover credentials. A solution to protect an extension from malicious manipulation is to provide extension certification via a trusted third party certifying authority \cite{VarshneyBS18}. Other solutions consist of notifying the user of undesired changes in the \gls{dom} or providing permissions where the user can block and allow the extension's access \cite{VarshneyBS18}. Protecting the implemented browser extension is one important task for future work. \\

Currently, the user can only select from already saved credentials. To expand usability, the extension could also send newly created credentials back to the application. The application then encrypts, and stores received data. The user is not constrained to store credentials beforehand in the app. 

\subsection{Improvements of the Android application}
Another future work task can be the integration of a master password when the mobile device does not support fingerprint scanning. Many phones are not able to authenticate the user via fingerprint because of a missing sensor. Hence, a different possibility for authentication must be provided. This can be done by asking the user to set a master password at application start. The master password can be hashed by using a cryptographic hashing function, such as bcrypt. The generated hash of the master password can be stored into a separate table of the existing database. Whenever the user enters the master password, the hash is generated and compared to the hash stored in the database. If they match, access is granted. Therefore, the master password is not stored on the device. \\

As stated in Section \ref{introduction}, users tend to use insecure passwords and re-use them for multiple accounts. By implementing a password generator, the process of password creation can be simplified. Whenever a user creates a new account, the option of generating a password can be provided. This mechanism helps with satisfying security guidelines such as using a password that consists of more than ten characters, special characters, letters, and numbers.

