\section{Future Work} \label{futurework}
This section presents ideas that could be actualized in the near future once the necessary technology has progressed. Also, we discuss improvements regarding usability, convenience, and security of the Bluetooth connection, the extension, and the application. These features are not part of the scope of this project and therefore have not been implemented further since they may be quite time-consuming to realize or not possible with start-of-the-art technology.

\subsubsection*{Improvements and future work on the Bluetooth connection}
One possible future implementation can be the encryption of the \gls{ble} connection to provide further security. By securing the communication channel with own encryption, a more secure environment to exchange credentials can be provided. Therefore, the implementation of classic Bluetooth can be spared by making the use of \gls{ble} more secure. The encryption would take place before the \gls{tk} is exchanged. This prevents the \gls{tk} from being retrieved. As mentioned in Section \ref{bluetooth}, with \gls{ble} the risk of extracting data  via packet sniffers  exists. By encrypting the connection, this risk can be reduced as well. \\

Another consideration is switching from a \gls{ble} to a classic Bluetooth connection. This depends on Google's Web Bluetooth API. Currently only \gls{ble} is supported. As discussed in Section \ref{bluetooth}, \gls{ble} can lead to a vulnerable communication channel when keys can be retrieved during the pairing process. Also, the Web Bluetooth API is still experimental, and the specification is not final yet. This leaves the possibility open for further development and eventually the support for classic Bluetooth.
 
\subsubsection*{Improvements and future work of the Chrome extension}
In \cite{VarshneyBS18} malicious extensions are discussed and how a Chrome extension can become vulnerable to various attacks. Malicious extensions can perform attacks, where sensitive data is phished from the user or keystrokes are recorded to recover credentials. A solution to protect an extension from malicious manipulation is to provide extension certification via a trusted thrid party certifying authority \cite{VarshneyBS18}. Other solutions consist of notifying the user of undesired changes in the \gls{dom} or providing permissions where the user can block and allow the extension's access \cite{VarshneyBS18}. Protecting the implemented browser extension is one important task for future work. \\

Currently, the user can only select from already saved credentials. To expand usability, the extension could also send newly created credentials back to the application. The application then encrypts, and stores received data. The user is not constrained to store credentials beforehand onto the app. 

\subsubsection*{Improvements and future work of the Android application}
Another future work task can be the integration of a master password when the mobile device does not support fingerprint scanning. Many phones are not able to authenticate the via fingerprint because of a missing sensor. Hence, a different possibility for authentication must be provided. This can be done by asking the user to set a master password by application start. The master password can be stored into a separate table of the existing database by encrypting it using \gls{pbkdf2}. \\
The master password is chosen by the user. As discussed in Section \ref{arch_encryption}, we strive to make the encryption independent from the user's input. In \cite{percival2009stronger} it is stated that \gls{pbkdf2} does not require much memory making it less expensive to perform brute force attacks. As a result of this paper \cite{DurmuthGKPYZ12} it was possible to recover passwords the with graphics cards. If the master password can be recovered by brute forcing it, the application can be vulnerable. Scrypt uses more CPU time and is stronger than bcrypt and \gls{pbkdf2} \cite{percival2009stronger}. The table containing the master password can be additionally encrypted with scrypt to provide higher security. \\

As stated in Section \ref{introduction}, users tend to use insecure passwords and re-use them for multiple accounts. By implementing a password generator the process of password creation can be simplified. Whenever a user creates a new account, the option of generating a password can be provided. This mechanism helps with satisfying security guidelines such as using a password that consists of more than 10 characters, special characters, letters, and numbers.

