\section{Limitations with Root Privilege} \label{limitations}
In this section, we discuss encountered limitations regarding the security of the mobile device. \\
When the device is rooted, the security of system resources and sensitive data is drastically reduced. As stated in \cite{DBLP:conf/acisp/ZhangWJWL14}, the Android permission system is one of the most important security mechanism for protecting critical system resources on Android phones. \\
Before an application can access resources, it needs certain permissions which are granted by the Android permission system. The application asks for permission with check functions, and in case of approval the API returns \texttt{PERMISSION\_GRANTED}, or in case of denial, it returns \texttt{PERMISSION\_DENIAL}.

Root privilege on an Android phone is the highest privilege in user mode. Hence, a rooted device does not require any further permissions from the Android system. Android's Shared Preferences and SQLite databases become accessible \cite{DBLP:conf/nsyss/ShezanAI17}. The rooted device gains access to sensitive data or hardware interfaces. This particularly leads to high risk in security when storing sensitive data. As a result, the vulnerable application cannot provide secure credential storage. \\

Not only is a rooted device a threat to secure data storage, but also when the device is unrooted again. In this case, the permission system can still be bypassed by tampering with the packages file or the apk files. These are both important files of the permission system \cite{DBLP:conf/acisp/ZhangWJWL14}. Even after a phone has been unrooted, malware still has the possibility of exploiting system resources. \\ \cite{DBLP:conf/acisp/ZhangWJWL14} discusses that root privilege can add a long-term security threat to the permissions system even if the device has no root privileges anymore.