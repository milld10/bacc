\section{Related Work} \label{relwork}

There exist various approaches to store and manage credentials. Passwords are most commonly used to authenticate a user, despite the fact they have caused many security issues \cite{JacommeK18}, \cite{WeakPasswords}. We will discuss some existing solutions and their limitations in security and convenience. \\

First, we will look into web-based password managers. Web-based password managers work in web browsers and manage credentials for web applications. It also automates the process of inserting credentials into the login fields. This frees the user from remembering a different password for each web application and the process of typing the credentials into the fields \cite{LiHAS14}. \\
One popular and widely used web-based password manager is LastPass. The user can access their passwords through the LastPass extension or also through the LastPass website. The attack surface of web-based password managers can be broken into different areas including classic web vulnerabilities and authorization vulnerabilities \cite{LiHAS14}. A classic web vulnerability results from an insufficient understanding of the web security model. This vulnerability can lead to cross-site request forgery (CSRF). Authorization vulnerabilities can arise when sharing and updating credentials are not fully authorized. \\
Research \cite{LiHAS14} has shown, the popular web-based password manager LastPass authenticates a user using cookies. This results in a vulnerability that can lead to CSRF attacks. Exploiting the CSRF attack, an attacker can obtain a user's encrypted database where attacks such as a denial of service (DoS) attack can be executed. \\
As discussed in \cite{AndroidPhishing}, the Android application of LastPass, as well as other well-known Android password managers, are prone to exploitation through phishing attacks. The user must interact with the provided web backend in the Android version of LastPass. Through malicious and tampered applications it was possible to mislead the user and retrieve their credentials. \\

A different approach to store and access sensitive data is using multi-factor authentication. This technique increases security by combining multiple authentication factors. It introduces an additional device into the authentication process such as a mobile device or an authentication token \cite{JacommeK18}. One widely-used protocol is FIDO's Universal 2nd Factor (U2F) protocol. It requires additional hardware in the form of a USB that is connected to the computer in use. An example is \textit{YubiKey} \cite{Yubikey}. \\
With this USB token, keys are generated and stored securely. Also, cryptographic computations are done using those keys. The following steps perform the authentication. First, user credentials are sent to the server where the user wants to login to. A challenge is generated by the server and sent back to the user's computer. The user confirms by pressing the button on the USB token. Lastly, the login process is verified by sending the signature to the server \cite{JacommeK18}. \\
The implementation of U2F through YubiKey is considered to be secure against attacks such as phishing, session hijacking, and man-in-the-middle. According to \cite{JacommeK18}, YubiKey may lead to different problems regarding feedback for the success of the button press. Also, protection against malware cannot be provided. As a solution, the storage of keys in a \gls{tee} is suggested such as on a mobile phone.

However, web-based approaches serve as a third-party between user and web application, and multi-factor authentication requires additional hardware compromising convenience. \\
For this project, we strive to reduce external dependencies altogether. We will handle an unexplored approach of sending credentials from the mobile device through a Bluetooth connection to sign in while encryption keys lie in the \gls{tee} of the mobile phone.
