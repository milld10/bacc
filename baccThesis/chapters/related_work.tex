\section{Related Work} \label{relwork}
Passwords are most commonly used to authenticate a user, despite the fact they have caused many security issues. For example, \cite{WeakPasswords} discusses how weak passwords and reuse can lead to data breaches. We will discuss some existing solutions and their limitations in security and convenience. \\

First, we will look into web-based password managers. Research \cite{LiHAS14} states that web-based password managers work in web browsers and manage credentials for web applications. They also automate the process of inserting credentials into the login forms. This method frees the user from remembering different passwords for web applications and the process of typing the credentials into the forms. \\
One popular and widely used web-based password manager is LastPass. The user can access their passwords through the LastPass extension or the LastPass website. However, password managers are not immune to attacks. For example, \cite{LiHAS14} has shown, LastPass authenticates a user using cookies. This results in a vulnerability that can lead to various attacks, for example, \gls{csrf} and \gls{dos} attacks. Additionally, as discussed in \cite{AndroidPhishing}, the Android application of LastPass, as well as other well-known Android password managers, are prone to exploitation through phishing attacks. The user interacts with the provided web backend in the Android version of LastPass, and through malicious and tampered applications it was possible to mislead the user and retrieve their credentials. \\

A different approach to store and access sensitive data is using multi-factor authentication. As discussed in \cite{JacommeK18}, this technique increases security by combining multiple authentication factors. It introduces an additional device into the authentication process, such as a mobile device or an authentication token. One widely-used protocol is FIDO's \gls{u2f} protocol. It requires additional hardware in the form of a USB that is connected to the computer in use. An example is \textit{YubiKey} \cite{Yubikey}. \\
With this USB token, keys are generated and stored securely. Also, cryptographic computations are done using those keys. Research \cite{JacommeK18} lists the following steps to perform the authentication. First, user credentials are sent to the server where the user wants to login to. A challenge is generated by the server and sent back to the user's computer. The user confirms by pressing the button on the USB token. Lastly, the login process is verified by sending the signature to the server. \\
The implementation of \gls{u2f} through YubiKey is considered to be secure against attacks, such as phishing, session hijacking, and \gls{mitm}. Still, according to \cite{JacommeK18}, YubiKey may lead to different problems regarding feedback for the success of the button press. Also, protection against malware cannot be provided. As a solution, the storage of keys in a \gls{tee}, such as on a mobile phone, is suggested (cf. Chapter \ref{arch_keystore}).

Web-based approaches serve as a third-party between a user and a web application, and multi-factor authentication requires additional hardware compromising convenience. \\
For this project, we strive to reduce external dependencies altogether. We will handle an unexplored approach of sending credentials from the mobile device through a Bluetooth connection to sign in while encryption keys lie in the \gls{tee} of the mobile phone.
