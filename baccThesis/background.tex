\section{Background}
\label{background}
This Section briefly discusses the structure and background of technologies used in this thesis.

\subsubsection*{Structure of Android Applications}
As stated in \cite{liu2011research}, the Android operating system is open source and based on the Linux kernel. Android was launched by Google. For every application, an own Linux process is running, and the app's code is isolated from the others. \cite{AndroidFundamentals} \\
An Android application consists of essential components. The four different app components are activities, services, broadcast receivers, and content providers. \cite{liu2011research}

Activities are single screen providing an interface to interact with the user. It shows what is on the screen for the user to see. Activities are linked together to form one application, but they still are independent of each other. If a new activity is started, the previous one stops and is preserved in a stack. When launching the app, an activity called \textit{main activity} is typically presented to the user. \\
A service does not provide an interface. Rather it runs in the background performing long-running operations. An activity can either start a service or bind to it for interaction. \cite{AndroidFundamentals} The BLE service implemented in this project represents such a service component. \\
Broadcast receivers receive and deliver system-wide broadcast messages. For example, a scheduled alarm from the application. Hence, the application does not need to be running for the alarm to go off. Broadcasts can also be sent from the system, for example when the battery is low. \\
Content providers provide mechanisms for applications to share data across the system. An SQLite database is an example where data can be stored through a content provider. \cite{AndroidFundamentals} In this project we use the content provider \textit{AndroidKeystore} to store user credentials. 

For the system to be able to start app components, a file named \textit{AndroidManifest} must exist. The manifest is an XML file that holds the application's components as well as permissions. \cite{AndroidFundamentals} 

\subsubsection*{Structure of Chrome Extensions}
Extensions provide a customized browsing experience for the Google Chrome browser. Similar to the Android applications, they consist of different components as well. Components of an extension are background scripts, content scripts, options page, and UI elements written in HTML, CSS, and JavaScript. \cite{Extensions} Content scripts are used to access the document object model (DOM) and are injected into the loaded web page. \cite{DBLP:conf/ndss/LiuZYC12} \\
Extensions need a manifest file. The manifest consists of the extension name, the version, a description, and a manifest version. To utilize the extension in the Chrome browser, one folder containing all files is loaded on the page \texttt{chrome://extensions}. \cite{Extensions}


\subsubsection*{Bluetooth Low Energy}
Bluetooth Low Energy (BLE), like classic Bluetooth, allows a wireless short-range communication between devices. BLE was introduced to provide communication with a much lower power consumption than classic Bluetooth. \cite{DBLP:journals/sensors/GomezOP12}. Since BLE has been derived from the classic Bluetooth, the protocol stack of both technologies consist of a Controller and a Host. As stated in \cite{overview and eval ble}, the Controller consists of the Physical and Link Layer, while the Host manages the upper layers. The upper layers include the Attribute Protocol (ATT), the Generic Attribute Protocoll (GATT), and the Generic Access Protocol (GAP). \\
However, the protocol stack of BLE is not compatible with the one of classic Bluetooth. This means if a device only implements BLE, it cannot communicate with another device that implements classic Bluetooth. \cite{DBLP:journals/sensors/GomezOP12}

BLE can only transmit and receive a single packet. \cite{DBLP:conf/woot/Ryan13} The size of BLE message can be 47 bytes at most, while the maximum message size of classic Bluetooth is 358 bytes. \cite{DBLP:journals/sensors/GomezOP12} Therefore, BLE is designed to send small packets containing less information than with classic Bluetooth.

The ATT describes the communcation of two connected devices. The application uses the GATT layer for communcation of characteristics. Characteristics contain a value and properties and permissions, and they can be stored and passed. \cite{BTGatt} A universal unique identifier (UUID) defines the type of the characteristic. \\
A connected device can either have the role of the GATT server or the GATT client. \cite{TIGatt}   The GATT server has the characteristics and the GATT client can read or write to characteristics. \\
The GAP is at the highest level of the BLE stack. Without GAP, discovery of devices and services, establishing the connection, and security would not be possible. It defines device roles, modes and procedures for these tasks. \cite{DBLP:journals/sensors/GomezOP12}


\subsubsection*{Chrome Web Bluetooth API}
The Web Bluetooth API makes it possible for websites to communicate with Bluetooth devices. Bluetooth devices must be able to establish a Bluetooth Low Energy (BLE) connection for websites to interact with them. Also, the must implement the BLE peripheral role, since the Web Bluetooth API acts the central role. \cite{BTAPIMozilla} \\
Websites can connect to user-selected Bluetooth devices and can read and write to characteristics. The Web Bluetooth API searches for nearby BLE devices and connects to the Bluetooth remote GATT Server. This, in return, holds the advertised service and characteristics. To read a characteristic, the API first gets the Primary GATT Service with the method \texttt{getPrimaryService()}. With the method \texttt{getCharacteristic()}, the characteristic is retrieved. To read the values, the method \texttt{readValue()} is used. For example, this can be the battery level of the device or similar information. \cite{WebBTAPI}